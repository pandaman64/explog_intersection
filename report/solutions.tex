\section{解の種類}

\subsection{不動点解}
	$y = a^x$と$y = \loga x$はたがいに逆関数であるから、$a^x = x$を満たすような$x$、すなわち指数関数や対数関数の不動点は、$a^x = \loga x( = x)$となってこれらの交点の一つと分かる。
	よって、このような$x$を不動点解と呼ぶこととする。
	また、$y = x,y = a^x$の増減から、端点を除いて$a > 1$では二つの不動点解が存在し、$0 < a < 1$では一つの不動点解が存在することが分かる。$a > 1$のときの、不動点解の大きい方を不動点大解、小さい方を不動点小解と呼ぶこととする。
	不動点解の存在する範囲を求めよう。
	\begin{theorem}
	\label{th:fixed_solutions}
		不動点解は存在するような$a$の範囲は$0 < a < e^\frac{1}{e}$である。
	\end{theorem}
	\begin{proof} \mbox{}\\
		\begin{enumerate}
			\item $0 < a <= 1$のとき
			
				$y = x$と$y = a^x$はどちらも単調に変化し、増減から$0 < x < 1$の範囲で一つ交点を持つので、不動点解を一つ持つ。
			\item $a > 1$のとき
			
				$y = a^x$のグラフを考えると、$a$を小さくしていくにつれてこれは直線$y = x$に近づき、いつか一点で接する。この接点において、
				\begin{align*}
					a^x &= x \quad (\text{二つのグラフの交点である}) \\
					a^x\ln{a} &= 1 \quad (\text{接線の傾き1})
				\end{align*}
				が成り立つから、これを解くと
				\begin{align*}
					a &= e^\frac{1}{e} \\
					x &= e 
				\end{align*}
				が得られる。
		\end{enumerate}
		これらより、\thref{th:fixed_solutions}が証明された。
	\end{proof}

