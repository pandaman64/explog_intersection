\section{今後の課題}

	指数関数と対数関数の交点についての$a$と$x$の陰関数は全て見つけることが出来た。しかし、それら見つけた陰関数のうち、$a=x^{\frac{1}{x}}$は$x$の関数であり、まだ全ての$x$を底$a$の関数で表わすという目標は達成できていない。
	調べたところでは、ランベルトの$W$関数を用いれば、それも可能になると思われるが、まだ学習が進んでいないのでよく分からない。今後自分が数学のより深い理解を得た時の課題としたい。
	
\section{感想}

	最初にこのレポートを書いた時には、不動点についての$a=x^{\frac{1}{x}}$という関係式を見つけることしかできず、不満の残るものであった。
	しかし、偶然テトレーションと出会い、さらにそれを利用した論文を読むことによって、目の前の道が大きく開けた気がした。まさに巨人の肩の上に乗った小人のような気分だった。
	
	先人達の努力とその結果として作り上げられた人類の共有知の偉大さを感じ、いつか自分もそれに貢献できるようになりたいと強く思った。
	
\begin{thebibliography}{9}
	\bibitem{knobel} R. Arthur Knoebel, ``Exponentials Reiterated,'' The American Mathematical Monthly, Vol. 88, No. 4 (Apr., 1981), pp. 235-252.
	\bibitem{balm} balm, http://balm.web.fc2.com/math79.html
\end{thebibliography}
