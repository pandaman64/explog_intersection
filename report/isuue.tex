\section{今後の課題}

	上記のように、指数関数と対数関数の交点についての$a$と$x$の陰関数は全て見つけることが出来た。しかし、それら見つけた陰関数のうち、$a=x^{\frac{1}{x}}$は$x$の関数であり、まだ全ての$x$を底$a$の関数で表わすという目標は達成できていない。
	ランベルトの$W$関数を用いれば、全ての解$x$を底$a$で表わすことも可能になると予想しているが、まだ理解が進んでいないのでこのアプローチが上手くいくかどうかは不明である。これは、今後自分が数学のより深い理解を得た時の課題としたい。
	
\section{おわりに}

	最初にこのレポートを書いた時には、不動点についての$a=x^{\frac{1}{x}}$という関係式を見つけることしかできず、不満の残るものであった。
	しかし、偶然テトレーションと出会い、さらにその性質を解説した論文\cite{knoebel}を読むことによって、目の前の道が大きく開けた気がした。まさに巨人の肩の上に乗った小人のような気分だった。
	
	先人たちの努力の成果である人類の共有知は偉大であり、その共有知に簡単にアクセスできるということは、本当に素晴らしいと感じた。いつか自分も人類の知に貢献できるような人間になりたいと強く思った。
	
\begin{thebibliography}{9}
	\bibitem{knoebel} R. Arthur Knoebel, ``Exponentials Reiterated,'' The American Mathematical Monthly, Vol. 88, No. 4 (Apr., 1981), pp. 235-252.
	\bibitem{balm} balm, \url{http://balm.web.fc2.com/math79.html}
\end{thebibliography}

このレポートおよびレポート製作用のデータ、プログラムなどは、\url{https://github.com/pandaman64/explog_intersection}で公開している。