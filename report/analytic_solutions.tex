\section{解析解}

数値解析によって解の値は求められた。では、これらの値はいったいどのような数式であらわされるのだろうか?

指数関数と対数関数のグラフにおいて、
\begin{align*}
	\loga x &= a^x \\
⇔	a^{a^x}   &= x
\end{align*}
だから、$f(a,x) = a^{a^x} - x$として、
\begin{align*}
	f(a,x) &= 0 \quad \text{となるような$a$や$x$についての関数} \\
		 a &= g(x) \\
		 x &= h(a) \\
\end{align*}
を求めるということが目標となる。このような関数$a = g(x)$や$x = h(a)$は$f(a,x) = 0$の陰関数と呼ばれ、この場合いくつか存在する。

\subsection{不動点解}
	まずは、不動点解について調べてみよう。指数関数の不動点$\xf$において、
	\begin{equation*}
		a^{\xf} = \xf
	\end{equation*}
	であるから、$x$についての関数$a = x^{\frac{1}{x}}$は$f(a,x) = 0$の陰関数である。
	
	つまり、不動点解においては、解から底を求めることが可能になったと言える。
	
	では、逆に底から解を導くような陰関数はどのように表わされるだろうか。それを述べるには少し特殊な演算を導入する必要がある。
	
\subsection{テトレーション}
	テトレーションとは、加算や乗算、累乗の延長線上に定義される演算だ。これらの演算は、
	\begin{align*}
		a + b &= a +  \underbrace{1 + 1 + \cdots + 1}_b \\
		a \times b &= \underbrace{a + a + \cdots + a}_b \\
		a^b   &= \underbrace{a \times a \times \cdots \times a}_b
	\end{align*}
	というように、繰り返しの形で表わされる演算であり、ハイパー演算と呼ばれている。テトレーションは加算、乗算に次ぐ三番目のハイパー演算である累乗の繰り返し、すなわち
	\begin{equation*}
		^b a = \underbrace{a ^{a ^{\cdot ^{\cdot ^a}}}}_b
	\end{equation*}
	と定義され、ハイパー関数の内で4番目であることからテトレーション(\emph{tetra}tion)と呼ばれている。他にも超累乗(hyperpower)、累乗列(powertower)と呼ばれることもあるが、ここではテトレーションで呼び方を統一する。
	また、テトレーション$^b a$の$a$をテトレーションの底と呼び、$b$を高さと呼んでいる。
	
	それでは、さっそくテトレーションを活用しよう。テトレーション$^n a$について、高さ$n$を無限に大きくしていくと、底$a$の値によっては$^n a$はある値に収束する。この収束値を高さ無限大のテトレーション$^\infty a$とあらわすこととする。
	すなわち、
	\begin{equation*}
		^\infty a = \lim_{h \to \infty} {^h a}
	\end{equation*}
	とする。テトレーションの定義から、$^{h+1} a = a^{^h a}$なので、
	\begin{align*}
		a ^{^\infty a} &= {^\infty a} \quad \text{であって、$x = {^\infty a}$とおけば} \\
		a^x &= x
	\end{align*}
	である。ここから、$a$についての関数$x = {^\infty a}$は、不動点解についての$f(a,x) = 0$の陰関数表示であることが分かる。テトレーションの導入によって、指数関数と対数関数の交点を底であらわすことができるようになった。
	
\subsection{$^\infty a$の収束}
	それでは、高さ無限大のテトレーション$^\infty a$がどのような$a$について収束するかを調べよう。
	
	ここで、$x = {^\infty a}$は$a = x^\frac{1}{x}$の逆関数であり、$a = x^\frac{1}{x}$の値域が$0 < a \leq e^\frac{1}{e}$であることから、この範囲で$^\infty a$は収束すると想定される。
	この想定は、\thref{th:fixed_solutions}で示された結果とも良く対応しているように見える。
	
	しかし、実際には\thref{th:conjugate_solutions}で示されたように、共役解の影響が出てしまうので、この範囲全体では$^\infty a$は収束しないのだ。収束範囲について以下に示す。
	\begin{theorem}
	\label{th:tetration_convergence}
		$^\infty a$は$\frac{1}{e^e} < a \leq e^\frac{1}{e}$のとき、存在する
	\end{theorem}
	
	\begin{lemma}
		$1 \leq a \leq e^\frac{1}{e}$のとき、$\lim_{n \to \infty} {^n a}$は収束する。
	\end{lemma}
	\begin{proof}
	
		$a \geq 1$より、$^{n+1} a = a ^{^n a} \geq {^n a}$であるから、数列$\{^n a\}$は、単調増加する。
		また、$1 \leq z \leq e$である$z$をおくと、
		\begin{align*}
			1 \leq a^z &\leq (e^\frac{1}{e})^e \\
					 &= e
		\end{align*}
		であるから、$1 \leq a \leq e^\frac{1}{e} < e$から、帰納的に、
		\begin{equation*}
			1 \leq {^n a} \leq e
		\end{equation*}
		であるといえる。これらより、数列$\{^n a\}$は単調増加し、上に有界な数列となるので、単調収束定理から、数列$\{^n a\}$が収束する。よって、
		\begin{equation*}
			^\infty a = \lim_{n \to \infty} {^n a}
		\end{equation*}
		が存在する。
	\end{proof}
	
	\begin{lemma}
		数列$A = \{^n a\}$の奇数番目だけを取り出した数列を$B = {B_n}$、偶数番目だけを取り出した数列を$C = {C_n}$とする。
		
		$0 < a < 1$のとき、数列$B$および$C$は収束する。
	\end{lemma}
	\begin{proof}
	
		$B_n$と$C_n$の定義から、
		\begin{align*}
			B_n = {^{2n-1} a} \\
			C_n = {^{2n} a}
		\end{align*}
		であり、累乗は二回適用したら大小関係を保つので、数列$B$および$C$はそれぞれ単調に変化することが分かる。ここで、$0 < z < 1$である$z$をおけば、$0 < a^z < 1$であるから、やはり帰納的に、
		\begin{align*}
			0 &<& {^n a} &< 1 \quad \text{すなわち} \\
			0 &<& b_n    &< 1 \\
			0 &<& c_n    &< 1
		\end{align*}
		
		したがって、数列$B$および$C$は単調に変化し有界であるから、それぞれ収束する。
	\end{proof}

	\begin{theorem}
	\label{th:tetration_implicit_function}
		$x = B_{\infty}$および$x = C_{\infty}$は$f(a,x)=0$の陰関数である。
	\end{theorem}
	
	\begin{lemma}
		$0 < a < \frac{1}{e^e}$のとき、$\lim_{n \to \infty} B_n \neq \lim_{n \to \infty} C_n$
	\end{lemma}
	\begin{proof}
	
		$0 < z_1 < \frac{1}{e}$である定数$z_1$と、$\frac{1}{e} < z_2 < 1$である定数$z_2$をおく。
		
		$0 < a < \frac{1}{e^e} < 1$であるから、
		\begin{align*}
			a^\frac{1}{e} &< a^{z_1} &< a^0 \\
						0 &< a^{z_2} &< \frac{1}{e^e}^{z_2}
		\end{align*}
		すなわち、
		\begin{align*}
			\frac{1}{e} &< a^{z_1} &< 1 \\
					  0 &< a^{z_2} &< \frac{1}{e}
		\end{align*}
		もう一回同じ操作を用いると、
		\begin{align*}
					  0 &< a^{a^{z_1}} &< \frac{1}{e} \\
			\frac{1}{e} &< a^{a^{z_2}} &< 1
		\end{align*}
		となる。$0 < B_1 = a < \frac{1}{e}$から、$\frac{1}{e} < C_1 < 1$が導かれ、したがって帰納的に、
		\begin{align*}
					  0 &< B_n &< \frac{1}{e} \\
			\frac{1}{e} &< C_n &< 1
		\end{align*}
		といえる。
		
		したがって、$\lim_{n \to \infty} B_n = \lim_{n \to \infty} C_n$ならば、これらは$\frac{1}{e}$に収束することになるが、\thref{th:tetration_implicit_function}から、この範囲で$x = \frac{1}{e}$が$f(a,x) = 0$をみたすことは無いから、
		結局$\lim_{n \to \infty} B_n \neq \lim_{n \to \infty} C_n$といえる。
	\end{proof}
	
	