\section{解析解}

数値解析によって解の値は求められた。では、これらの値はいったいどのような数式であらわされるのだろうか?

指数関数と対数関数のグラフにおいて、
\begin{align*}
	\loga x &= a^x \\
⇔	a^{a^x}   &= x
\end{align*}
だから、$f(a,x) = a^{a^x} - x$として、
\begin{align*}
	f(a,x) &= 0 \quad \text{となるような$a$や$x$についての関数} \\
		 a &= g(x) \\
		 x &= h(a) \\
\end{align*}
を求めるということが目標となる。このような関数$a = g(x)$や$x = h(a)$は$f(a,x) = 0$の陰関数と呼ばれ、この場合いくつか存在する。

\subsection{不動点解}
	まずは、不動点解について調べてみよう。指数関数の不動点$\xf$において、
	\begin{equation*}
		a^{\xf} = \xf
	\end{equation*}
	であるから、$x$についての関数$a = x^{\frac{1}{x}}$は$f(a,x) = 0$の陰関数である。
	
	つまり、不動点解においては、解から底を求めることが可能になったと言える。
	
	では、逆に底から解を導くような陰関数はどのように表わされるだろうか。それを述べるには少し特殊な演算を導入する必要がある。
	
\subsection{テトレーション}
	テトレーションとは、加算や乗算、累乗の延長線上に定義される演算だ。これらの演算は、
	\begin{align*}
		a + b &= a +  \underbrace{1 + 1 + \cdots + 1}_b \\
		a \times b &= \underbrace{a + a + \cdots + a}_b \\
		a^b   &= \underbrace{a \times a \times \cdots \times a}_b
	\end{align*}
	というように、繰り返しの形で表わされる演算であり、ハイパー演算と呼ばれている。テトレーションは加算、乗算に次ぐ三番目のハイパー演算である累乗の繰り返し、すなわち
	\begin{equation*}
		^b a = \underbrace{a ^{a ^{\cdot ^{\cdot ^a}}}}_b
	\end{equation*}
	と定義され、ハイパー関数の内で4番目であることからテトレーション(\emph{tetra}tion)と呼ばれている。他にも超累乗(hyperpower)、累乗列(powertower)と呼ばれることもあるが、ここではテトレーションで呼び方を統一する。
	また、テトレーション$^b a$の$a$をテトレーションの底と呼び、$b$を高さと呼んでいる。
	
	それでは、さっそくテトレーションを活用しよう。テトレーション$^n a$について、高さ$n$を無限に大きくしていくと、底$a$の値によっては$^n a$はある値に収束する。この収束値を高さ無限大のテトレーション$^\infty a$とあらわすこととする。
	すなわち、
	\begin{equation*}
		^\infty a = \lim_{h \to \infty} {^h a}
	\end{equation*}
	である。
	