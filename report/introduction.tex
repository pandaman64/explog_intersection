\section{はじめに}

	数学の逆関数の授業で、指数関数と対数関数のグラフを同じ座標平面上に書いたことがあった。
	異なる底の二つのグラフを書くと、底を小さくしていくにつれてそれらはだんだんと近づいていくように見えた。
	ならばこれらのグラフはいつか交点を持つだろうと考えたが、その時は交点が一体どのような形で表わすことが出来るか全く分からなかった。
	
	指数関数と対数関数のグラフの交点は底についてのどのような関数で表わされるのだろうか。このレポートはその成果をまとめたものである。

	このレポートでは、指数関数や対数関数の底を$a$で表わし、$y=a^x$や$y=\log_a x$の交点の$x$座標を解と呼ぶ。
	また、底$a$は$0 < a$の範囲で、解$x$も実数全体で考えている。$a=1$のときは、$x=1$でだけ二つの関数が共有点を持つとみなしている。
	
	はじめに、解の種類と、それぞれの性質について説明をする。
	次に、解を数値解析で求める方法を述べ、実際に数値解析によって得たグラフを示す。
	最後に、指数関数と対数関数の交点における底$a$や解$x$の関係式がどのようになるかを示し、テトレーションという演算を導入することで、解$x$を底$a$の関数で表わすことを目指す。
	