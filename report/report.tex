\documentclass[uplatex]{jsarticle}

\usepackage{amsmath}
\usepackage[dvipdfmx]{graphicx}

% from http://www.math.tohoku.ac.jp/~kuroki/LaTeX/howtolatex.html
\usepackage{amsthm}

\newtheorem{theorem}{定理}
\newtheorem{prop}[theorem]{命題}
\newtheorem{lemma}[theorem]{補題}
\newtheorem{cor}[theorem]{系}
\newtheorem{example}[theorem]{例}
\newtheorem{definition}[theorem]{定義}
\newtheorem{rem}[theorem]{注意}
\newtheorem{guide}[theorem]{参考}

\numberwithin{theorem}{section}  % 定理番号を「定理2.3」のように印刷
\numberwithin{equation}{section} % 式番号を「(3.5)」のように印刷

\usepackage{proof}
\renewcommand\proofname{\bf 証明}

\usepackage{cases}

\def\loga{\log_{a} }
\newcommand{\thref}[1]{定理(\ref{#1})}
% from http://calcurio.com/wordpress/?p=1632
\newcommand{\xf}{x_{\scalebox{0.5}{f}}}

\title{指数関数と対数関数のグラフの交点}
\author{pandaman64}
\date{\today}

\begin{document}

\maketitle

\section{はじめに}

	数学の逆関数の授業で、指数関数と対数関数のグラフを同じ座標平面上に書いたことがあった。
	異なる底の二つのグラフを書くと、底を小さくしていくにつれてそれらはだんだんと近づいていくように見えた。
	ならばこれらのグラフはいつか交点を持つだろうと考えたが、その時は交点が一体どのような形で表わすことが出来るか全く分からなかった。
	
	指数関数と対数関数のグラフの交点は底についてのどのような関数で表わされるのだろうか。このレポートはその成果をまとめたものである。

	このレポートでは、指数関数や対数関数の底を$a$で表わし、$y=a^x$や$y=\log_a x$の交点の$x$座標を解と呼ぶ。
	また、底$a$は$0 < a$の範囲で、解$x$も実数全体で考えている。$a=1$のときは、$x=1$でだけ二つの関数が共有点を持つとみなしている。
	
	はじめに、解の種類と、それぞれの性質について説明をする。
	次に、解を数値解析で求める方法を述べ、実際に数値解析によって得たグラフを示す。
	最後に、指数関数と対数関数の交点における底$a$や解$x$の関係式がどのようになるかを示し、テトレーションという演算を導入することで、解$x$を底$a$の関数で表わすことを目指す。
	
\section{解の種類}

\subsection{不動点解}
	$y = a^x$と$y = \loga x$はたがいに逆関数であるから、$a^x = x$を満たすような$x$、すなわち指数関数や対数関数の不動点は、$a^x = \loga x( = x)$となってこれらの交点の一つと分かる。
	このとき、$a = x^{\frac{1}{x}}$が成立する。
	このような$x$を不動点解と呼ぶこととする。
	また、$y = x,y = a^x$の増減から、端点を除いて$a > 1$では二つの不動点解が存在し、$0 < a < 1$では一つの不動点解が存在することが分かる。$a > 1$のときの、不動点解の大きい方を不動点大解、小さい方を不動点小解と呼ぶこととする。
	不動点解の存在する範囲を求めよう。
	\begin{theorem}
	\label{th:fixed_solutions}
		不動点解が存在するような$a$の範囲は$0 < a < e^\frac{1}{e}$である。
	\end{theorem}
	\begin{proof} \mbox{}\\
		$0 < a \leq 1$と$a > 1$のときで、場合分けをして考える。
		\begin{enumerate}
			\item $0 < a \leq 1$のとき
			
				$y = x$と$y = a^x$はどちらも単調に変化し、増減から$0 < x < 1$の範囲で一つ交点を持つので、不動点解を一つ持つ。
			\item $a > 1$のとき
			
				$y = a^x$のグラフを考えると、$a$を小さくしていくにつれてこれは直線$y = x$に近づき、いつか一点で接する。この接点において、
				\begin{align*}
					a^x &= x \quad (\text{二つのグラフの交点である}) \\
					a^x\ln{a} &= 1 \quad (\text{接線の傾き1})
				\end{align*}
				が成り立つから、これを解くと
				\begin{align*}
					a &= e^\frac{1}{e} \\
					x &= e 
				\end{align*}
				が得られる。
		\end{enumerate}
		これらより、\thref{th:fixed_solutions}が証明された。
	\end{proof}

\subsection{共役解}
	ところが、不動点以外でも指数関数と対数関数のグラフが交点を持つことがある。この解を共役解と定義し、小さい方から共役小解、共役大解とする。
	共役解が存在することは以下のように証明できる。
	\begin{theorem}
	\label{th:conjugate_solutions}
		$0 < a < \frac{1}{e^e}$のとき、共役解が存在する
	\end{theorem}
	\begin{proof} \mbox{}\\
		このとき、\thref{th:fixed_solutions}より、不動点解$\xf$が区間$(0,1)$に存在する。
		$f(x) = a^x - \log_a x$とすると、
		\begin{align*}
			f(1) = a^x > 0 \\
			f(\xf) = 0
		\end{align*}
		だから、$f'(\xf) < 0$ならば、区間$(\xf,1)$に$f(x) = 0$を満たす$x$が存在するといえる。
		\begin{align*}
			f'(\xf) = a^{\xf}\ln a - \frac{1}{\xf\ln a} &< 0 \\
									 a^{\xf}\xf\ln^2 a  &> 1 \quad (\ln a < 0\text{より}) \\
											  \ln^2 \xf &> 1 \quad (0 < a^{\xf} = \xf < 1\text{から}) \\
												0 < \xf &< \frac{1}{e} \\
												  0 < a &< \frac{1}{e^e}
		\end{align*}
		したがって、\thref{th:conjugate_solutions}が証明された。
	\end{proof}

\section{数値解}
	指数関数と対数関数の交点を簡単な底の関数で表わすのは難しそうだ。そこで、まずは解のグラフを書くことにした。
	グラフを書くために、交点の数値解を数値解析によって求めた。

\subsection{数値解析の方法}
	まずは、不動点解の数値解をニュートン法を用いて求めた。

	ニュートン法とは関数のグラフの傾きを用いて、数値解を求める方法だ。$y=a^x-x$として、適当な$x_0$をとり、
	\[
		x_{n+1} = x_n - \frac{y(x_n)}{y'(x_n)}
	\]
	という漸化式を繰り返し用いていくことで、$y=0$の$x$についての数値解が求められる。
	しかし、ニュートン法では初期値によって求まる解が変わり、また$y'(x) \sim 0$であるとき解への収束は不安定になってしまう。
	$y'(x) = 0$となるような$x$は$x = -\log_{a} \left(\ln{a}\right)$なので、この値から離れた初期値を採用することで二つの数値解を求めることが出来た。

	次に共役解の数値解を求めた。
	これは$y = a^{x} - \log_{a} x$とすると$y=0$の$x$についての数値解を調べることと同じだ。

	しかし、この関数についてニュートン法を適用することは難しい。なぜならば、$y' = a^{x}\ln{a} - \frac{1}{\ln{a}\ln{x}}$が複雑で、初期値を上手く決定できないからだ。

	したがって、二分法を用いてしらみつぶしに調べることにした。

	二分法は、中間値の定理を利用して数値解を求める方法だ。$y(m)y(n) < 0$であるような実数$m,n$について、中間値の定理から区間$(m,n)$に$y = 0$となる解が存在する。
	さらに、区間$(m,n)$内の実数$l = \frac{m+n}{2}$をとると、$g(m)g(l)$と$g(l)g(n)$の正負を調べることで、区間$(m,l)$と$(l,n)$のどちらに解が存在するかが分かる。
	この分割を十分大きな回数繰り返すと、区間が狭まっていき、最終的には数値解が求められる。

	しかし、この場合は$y$が符号を変えるような区間は事前に分からないが、それは解区間をしらみつぶしに調べることで解決した。
	共役解が$(0,1)$に存在することは慨形から分かるので、この区間を$2^{10}$程度の小区間に分割し、全てについて二分法を適用し、共役解を求めることが出来た。

	数値解のグラフを以下に示す。

\section{解析解}

数値解析によって解の値は求められた。では、これらの値はいったいどのような数式であらわされるのだろうか?

指数関数と対数関数のグラフにおいて、
\begin{align*}
	\loga x &= a^x \\
⇔	a^{a^x}   &= x
\end{align*}
だから、$f(a,x) = a^{a^x} - x$として、
\begin{align*}
	f(a,x) &= 0 \quad \text{となるような$a$や$x$についての関数} \\
		 a &= g(x) \\
		 x &= h(a) \\
\end{align*}
を求めるということが目標となる。このような関数$a = g(x)$や$x = h(a)$は$f(a,x) = 0$の陰関数と呼ばれ、この場合いくつか存在する。

\subsection{不動点解}
	まずは、不動点解について調べてみよう。指数関数の不動点$\xf$において、
	\begin{equation*}
		a^{\xf} = \xf
	\end{equation*}
	であるから、$x$についての関数$a = x^{\frac{1}{x}}$は$f(a,x) = 0$の陰関数である。
	
	つまり、不動点解においては、解から底を求めることが可能になったと言える。
	
	では、逆に底から解を導くような陰関数はどのように表わされるだろうか。それを述べるには少し特殊な演算を導入する必要がある。
	
\subsection{テトレーション}
	テトレーションとは、加算や乗算、累乗の延長線上に定義される演算だ。これらの演算は、
	\begin{align*}
		a + b &= a +  \underbrace{1 + 1 + \cdots + 1}_b \\
		a \times b &= \underbrace{a + a + \cdots + a}_b \\
		a^b   &= \underbrace{a \times a \times \cdots \times a}_b
	\end{align*}
	というように、繰り返しの形で表わされる演算であり、ハイパー演算と呼ばれている。テトレーションは加算、乗算に次ぐ三番目のハイパー演算である累乗の繰り返し、すなわち
	\begin{equation*}
		^b a = \underbrace{a ^{a ^{\cdot ^{\cdot ^a}}}}_b
	\end{equation*}
	と定義され、ハイパー関数の内で4番目であることからテトレーション(\emph{tetra}tion)と呼ばれている。他にも超累乗(hyperpower)、累乗列(powertower)と呼ばれることもあるが、ここではテトレーションで呼び方を統一する。
	また、テトレーション$^b a$の$a$をテトレーションの底と呼び、$b$を高さと呼んでいる。
	
	それでは、さっそくテトレーションを活用しよう。テトレーション$^n a$について、高さ$n$を無限に大きくしていくと、底$a$の値によっては$^n a$はある値に収束する。この収束値を高さ無限大のテトレーション$^\infty a$とあらわすこととする。
	すなわち、
	\begin{equation*}
		^\infty a = \lim_{h \to \infty} {^h a}
	\end{equation*}
	とする。テトレーションの定義から、$^{h+1} a = a^{^h a}$なので、
	\begin{align*}
		a ^{^\infty a} &= {^\infty a} \quad \text{であって、$x = {^\infty a}$とおけば} \\
		a^x &= x
	\end{align*}
	である。ここから、$a$についての関数$x = {^\infty a}$は、不動点解についての$f(a,x) = 0$の陰関数表示であることが分かる。テトレーションの導入によって、指数関数と対数関数の交点を底であらわすことができるようになった。
	
\subsection{$f(a,x)=0$の陰関数}
	それでは、高さ無限大のテトレーション$^\infty a$について調べることで、$f(a,x)=0$の陰関数を見つけ出そう。
	
	ここで、$x = {^\infty a}$は$a = x^\frac{1}{x}$の逆関数なので、${^\infty a}$が収束する範囲で関数は一致する。$a = x^\frac{1}{x}$の値域が$0 < a \leq e^\frac{1}{e}$であることから、${^\infty a}$が収束するのは少なくともこの範囲内である。
	
	まずは$a \geq 1$のとき、
	\begin{theorem}
		$1 \leq a \leq e^\frac{1}{e}$のとき、$\lim_{n \to \infty} {^n a}$は収束する。
	\end{theorem}
	\begin{proof}
	
		$a \geq 1$より、$^{n+1} a = a ^{^n a} \geq {^n a}$であるから、数列$\{^n a\}$は、単調増加する。
		また、$1 \leq {^n a} \leq e$と仮定すると、
		\begin{align*}
			1 \leq {^{n+1} a} &\leq (e^\frac{1}{e})^e \\
							  &= e
		\end{align*}
		であるから、帰納的に、
		\begin{equation*}
			1 \leq {^n a} \leq e
		\end{equation*}
		数列$\{^n a\}$は単調増加し、上に有界な数列となるので、単調収束定理から、数列$\{^n a\}$が収束する。
	\end{proof}
	
	次は、$0 < a < 1$のときだ。
	\begin{lemma}
		$0 < a < $とする。
		数列$\{^n a\}$の奇数番目と偶数番目を取り出した部分列$\{^{2n-1} a\},\{^{2n} a\}$はそれぞれ収束する。
	\end{lemma}
	\begin{proof}
	
		累乗は二回適用しても大小関係を保つので、それぞれの数列は単調に変化する。$0 < {^n a} < 1$と仮定すれば、
		\begin{align*}
			a^1<&{^{n+1} a}<a^0 \quad \text{より}\\
			0<&{^{n+1} a}<1 \\
		\end{align*}
		
		であるから、帰納的に$0 < {^n a} < 1$.
		したがって、数列$\{^{2n-1} a\},\{^{2n} a\}$はそれぞれ単調に変化し有界であるから、収束する。
	\end{proof}

	\[
		T(a) = \lim_{n \to \infty} {^n a},
		T_e = \lim_{n \to \infty} {^{2n} a},
		T_o = \lim_{n \to \infty} {^{2n - 1} a}
	\]と定義する。
	\begin{theorem}
	\label{th:tetration_implicit_function}
		$x = T_e(a)$および$x = T_o(a)$は$f(a,x)=0$の陰関数である。
	\end{theorem}
	
	\begin{theorem}
		$0 < a < \frac{1}{e^e}$のとき、$\displaystyle \lim_{n \to \infty} {^n a}$は発散する
	\end{theorem}
	\begin{proof}
		
		関数$y = x^x$を考えると、$x^x \geq (\frac{1}{e})^{\frac{1}{e}}$だから、$1 > a^a > \frac{1}{e}$
		また、$1 > {^{2n-1} a} > \frac{1}{e}$と仮定すれば、
		\begin{align*}
			0 < {^{2n} a} &< (\frac{1}{e^e})^{\frac{1}{e}} \\
						  &= \frac{1}{e} \quad \text{同様にして、} \\
			1 > {^{2n+1} a} &> \frac{1}{e}
		\end{align*}
		したがって、帰納的に$1 > {^{2n-1} a} > \frac{1}{e}$であるから、
		\begin{equation*}
			\lim_{n \to \infty} {^{2n - 1} a} \geq \frac{1}{e}
		\end{equation*}
		ところで、$a = x^{\frac{1}{x}}$は、$0 < a < \frac{1}{e^e}$のとき、$0 < x < \frac{1}{e}$であるから、$\displaystyle \lim_{n \to \infty} {^n a}$が収束するならば、その部分列についても
		\begin{equation*}
			\lim_{n \to \infty} {^{2n - 1} a} < \frac{1}{e}
		\end{equation*}
		となる。これは矛盾するので、$\lim_{n \to \infty} {^n a}$は発散する。
	\end{proof}
	
	\begin{theorem}
		$f(a,x)=0$の陰関数は、次のいずれかの関数と等価である。
		
		\[
		\left \{
			\begin{array}{l}
				a = x^{\frac{1}{e}} \\
				x = T(a) \\
				x = T_e(a) \\
				x = T_o(a)
			\end{array}
		\right.
		\]
	\end{theorem}
	\begin{proof}
		$f(a,x)$の$a$を定数として見た関数$f_a(x)$を考えると、
		\begin{align*}
			f_a'(x) &= {a^{a^x}}{a^x}{\ln^2 a} - 1 \\
			f_a''(x)&= {a^{a^x}}{a^x}{\ln^3 a}(a^x\ln a + 1)
		\end{align*}
		$a^x\ln a$は単調増加するから、
		
		\begin{itemize}
			\item $0 < a < 1$のとき
			
				$f_a''(x)$は単調減少
			\item $a > 1$のとき
			
				$f_a''(x) > 0$
		\end{itemize}
		したがって、$a^x\ln a + 1 = 0$のとき、$f_a'(x)$は極値
		\begin{align*}
			& \quad a^{-\frac{1}{\ln a}}(-1)(\ln a) - 1 \\
			&=		-\frac{\ln a}{e} - 1
		\end{align*}
		をとる。
		
		これまでの結果より、$0 < a < 1$のとき、$f_a(x) \leq -\frac{\ln a}{e} - 1$
		
	\end{proof}
	
\section{今後の課題}

	上記のように、指数関数と対数関数の交点についての$a$と$x$の陰関数は全て見つけることが出来た。しかし、それら見つけた陰関数のうち、$a=x^{\frac{1}{x}}$は$x$の関数であり、まだ全ての$x$を底$a$の関数で表わすという目標は達成できていない。
	ランベルトの$W$関数を用いれば、全ての解$x$を底$a$で表わすことも可能になると予想しているが、まだ理解が進んでいないのでこのアプローチが上手くいくかどうかは不明である。これは、今後自分が数学のより深い理解を得た時の課題としたい。
	
\section{おわりに}

	最初にこのレポートを書いた時には、不動点についての$a=x^{\frac{1}{x}}$という関係式を見つけることしかできず、不満の残るものであった。
	しかし、偶然テトレーションと出会い、さらにその性質を解説した論文\cite{knoebel}を読むことによって、目の前の道が大きく開けた気がした。まさに巨人の肩の上に乗った小人のような気分だった。
	
	先人たちの努力の成果である人類の共有知は偉大であり、その共有知に簡単にアクセスできるということは、本当に素晴らしいと感じた。いつか自分も人類の知に貢献できるような人間になりたいと強く思った。
	
\begin{thebibliography}{9}
	\bibitem{knoebel} R. Arthur Knoebel, ``Exponentials Reiterated,'' The American Mathematical Monthly, Vol. 88, No. 4 (Apr., 1981), pp. 235-252.
	\bibitem{balm} balm, \url{http://balm.web.fc2.com/math79.html}
\end{thebibliography}

このレポートおよびレポート製作用のデータ、プログラムなどは、\url{https://github.com/pandaman64/explog_intersection}で公開している。

\end{document}